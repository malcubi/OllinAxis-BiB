%$Header: /usr/local/ollincvs/Codes/OllinAxis-BiB/doc/usermanual.tex,v 1.2 2022/06/02 17:19:44 malcubi Exp $

%-----------------------------------------------------------------------
%
% filename = usermanual.tex
% 
%-----------------------------------------------------------------------

\documentclass[12pt]{article}

% Load packages for symbols and figures.

\usepackage{latexsym}
\usepackage{epsfig}
\usepackage{amsmath}
\usepackage{amssymb}
\usepackage{tikz}
\usetikzlibrary{shapes}

% Page settings.

\setlength{\textwidth}{170mm}
\setlength{\oddsidemargin}{-5mm}
\setlength{\evensidemargin}{-5mm}


%%%%%%%%%%%%%%%%%%%%%%%%%%
%%%   BEGIN DOCUMENT   %%%
%%%%%%%%%%%%%%%%%%%%%%%%%%

\begin{document}

% Reference equations with section number.

\renewcommand{\theequation}{\thesection.\arabic{equation}}

\parindent 0mm


%%%%%%%%%%%%%%%%%
%%%   TITLE   %%%
%%%%%%%%%%%%%%%%%

\title{Code \texttt{OllinAxis-BiB} \\ User's manual}

\author{Miguel Alcubierre \\
Instituto de Ciencias Nucleares, UNAM \\
malcubi@nucleares.unam.mx}

\date{June, 2022}

\maketitle

\tableofcontents


%%%%%%%%%%%%%%%%%%%%%%%%
%%%   INTRODUCTION   %%%
%%%%%%%%%%%%%%%%%%%%%%%%

\pagebreak

\section{Introduction}

This program solves the Einstein evolution equations in axial symmetry
using a curvilinear version of the BSSN formulation for the 3+1
evolution equations, with different types of matter and different
gauge conditions.

The main difference of this version of the code with respect to
previous ones is the fact that it uses box-in-box mesh refinement
(hence the BiB part of the name), and is parallelized with MPI.
Many parts of this code are based o a previous version written
by myself and José Manuel Torres.  An even older version was
written by Milton Ruiz.

Note: This manual is TERRIBLY INCOMPLETE!  I haven't has the
time to do it.  I will be adding a little bit more every now
and again. \\


%%%%%%%%%%%%%%%%%%%%
%%%   DOWNLOAD   %%%
%%%%%%%%%%%%%%%%%%%%

\section{Downloading the code}

If you are reading this it means you probably already downloaded the
code.  But if for some reason you need to download it here is how. \\

The easiest way to obtain the code is to download it from the CVS
server of the Ollin group (dulcinea.nucleares.unam.mx).  In order to
obtain a username and password for this sever you need to contact
Miguel Alcubierre (malcubi@nucleares.unam.mx). Once you have a
username and password you first need to log into the repository by
typing at the terminal: \\

\texttt{\footnotesize cvs -d
  :pserver:username@dulcinea.nucleares.unam.mx:/usr/local/ollincvs
  login} \\

where you should change the word ``username'' for your personal
username! You then need to type your password.  After this you must
download the code by typing: \\

\texttt{\footnotesize cvs -d
  :pserver:username@dulcinea.nucleares.unam.mx:/usr/local/ollincvs co
  Codes/OllinAxis-BiB} \\

This will create a directory \texttt{Codes} and inside it directory
\texttt{OllinAxis-BiB}. This is the main directory for the code.

\pagebreak


%%%%%%%%%%%%%%%%%%%%%%%
%%%   DIRECTORIES   %%%
%%%%%%%%%%%%%%%%%%%%%%%

\section{Directory structure}

The main directory for the code is \texttt{OllinAxis-BiB}.  There are
several sub-directories inside this main directory:

\begin{list}{}{
\setlength{\leftmargin}{40mm}
\setlength{\labelsep}{10mm}
\setlength{\labelwidth}{25mm}}

\item[\texttt{CVS}] Contains information about the CVS root and server (see
Sec.~\ref{sec:editing}).

\item[\texttt{doc}] Contains the tex and pdf files for this user's
  manual.

\item[\texttt{exe}] This directory is created at compile time and
  contains the executable file.  It also contains a copy of the
  parameter files found in directory \texttt{par} (see below).

\item[\texttt{fakempi}] Contains fake MPI routines so that the
  compiler won't complain if MPI is not installed.

\item[\texttt{gnuplot}] Contains a couple of simple gnuplot macros for
  visualization.

\item[\texttt{objs}] This directory is created at compile time and
  contains all the object and module files.

\item[\texttt{ollingraph}] Contains the visualization package
  ``ollingraph'' for convenient ``quick and dirty'' visualization (see
  Section~\ref{sec:ollingraph} below).

\item[\texttt{par}] Contains examples of parameter files (see
Section~\ref{sec:parfiles} below).

\item[\texttt{prl}] Contains perl scripts used at compile time to
  create the subroutines that manage parameters and arrays.

\item[\texttt{src}] Contains the source files for all the code routines.

\end{list}


The directory \texttt{src} is itself divided into a series of
sub-directories in order to better classify the different
routines. These sub-directories are:

\begin{list}{}{
\setlength{\leftmargin}{40mm}
\setlength{\labelsep}{10mm}
\setlength{\labelwidth}{25mm}}

\item[\texttt{CVS}] Contains information about the CVS root and server.

\item[\texttt{auto}] Contains \texttt{FORTRAN} files that are
  automatically generated at compile time by the perl scripts.  These
  files should not be edited manually!

\item[\texttt{base}] Contains the routines that control the basic
  execution of the code, including the parameter and array
  declarations, the parameter parser, the output routines, the main
  evolution controllers, and generic routines for calculating
  derivatives, dissipation, etc.  The code in fact starts execution at
  the routine \texttt{main.f90} contained in this directory.

\item[\texttt{elliptic}] Contains routines for solving elliptic
  equations for initial data and/or maximal slicing for example.

\item[\texttt{geometry}] Contains routines related to initial data,
  evolution and analysis of the spacetime geometric variables,
  including sources, gauge conditions, constraints, horizon finders,
  etc.

\item[\texttt{matter}] Contains routines related to the initial data,
  evolution and analysis of the different matter models, including a
  generic routine for calculating the basic matter variables, and
  routines for evolving scalar fields, electric fields, fluids, etc.

\end{list}

\vspace{3mm}


%%%%%%%%%%%%%%%%%%%%%%%%%%%%%%%%%
%%%   COMPILING AND RUNNING   %%%
%%%%%%%%%%%%%%%%%%%%%%%%%%%%%%%%%

The code is written in \texttt{FORTRAN 90} and is parallelized with
\texttt{MPI} (Message Passing Interface).  All subroutines are in
separate files inside the directory \texttt{src} and its
sub-directories.

\subsection{Compiling}
\label{sec:compiling}

To compile just move inside the \texttt{OllinSphere-BiB} directory and type: \\

\texttt{make} \\

This will first run some perl scripts that create a series of
automatically generated \texttt{FORTRAN} files that will be placed
inside the directory \texttt{src/auto}.  It will then compile all the
\texttt{FORTRAN} routines that it can find inside any of the
sub-directories of \texttt{src} (it will attempt to compile any file
with the extension \texttt{.f90}). \\

The resulting object files and \texttt{FORTRAN} module files will be
placed inside the sub-directory \texttt{objs}. The Makefile will then
create a directory \texttt{exe} and will place in it the final
executable file called \texttt{ollinaxis}.  It will also copy to this
directory all the sample parameter files inside the sub-directory
\texttt{par}, and the visualization package \texttt{ollingraph}. \\

Notice that at this time the Makefile can use the compilers
\texttt{g95}, \texttt{gfortran}, or the Intel compilers \texttt{ifc}
and \texttt{ifort}, and it will automatically check if they are
installed. If you have a different compiler then the Makefile will
have to be modified (hopefully it won't be very difficult). The code
will also attempt to find an \texttt{MPI} installation (it looks for
the command \texttt{mpif90}), and if it does not find it it will use
the fake routines inside the directory \texttt{fakempi}. \\

The Makefile has several other useful targets that can be listed by
typing: \texttt{make help}. \\


\subsection{Running}
\label{sec:running}

To run the code move into the directory \texttt{exe} and type: \\

\texttt{ollinsphere name.par} \\

Where \texttt{name.par} is the name of your parameter file (more on
parameter files below).  The code will then read data from the
parameter file silently and hopefully start producing some output to
the screen. The code will also create an output directory and will
write the data files to that directory. \\

For parallel runs using \texttt{MPI} one must use instead the command:
\\

\texttt{mpirun -np N ollinsphere name.par} \\

where \texttt{N} should be an integer number that specifies the number
of processors to be used. \\


%%%%%%%%%%%%%%%%%%%%%%%%%%%
%%%   PARAMETER FILES   %%%
%%%%%%%%%%%%%%%%%%%%%%%%%%%

\section{Parameter files}
\label{sec:parfiles}

At run time the code reads the parameter values from a parameter file
(parfile), with a name of the form \texttt{name.par}, that must be
specified in the command line after the executable: \\

\texttt{ollinaxis  name.par} \\

The data in this parameter file can be given in any order, using the
format: \\

\texttt{parameter = value} \\

Comments (anything after a \texttt{\#}) and blank lines are ignored.
Only one parameter is allowed per line, and only one value is allowed
per parameter, with the exception of the parameters \texttt{outvars0D}
and \texttt{outvars1D} that control which arrays get output and take
lists of arrays as values, for example: \\


There is in fact one other parameter that can also take multiple
values as input, it is the parameter \texttt{mattertype} that can
accept several types of matter at once (see Section~\ref{sec:matter}
below).\\

Parameters that do not appear in the parfile get the default values
given in the file \texttt{param.f90}.  Examples of parameter files can
be found in the subdirectory \texttt{par}. \\

IMPORTANT: Even though \texttt{FORTRAN} does not distinguish between
upper and lower case in variable names, the names of parameters are
handled as strings by the parameter parser, so lower and upper case
are in fact different.  The name of parameters in the parameter file
should then be identical to the way in which they appear in the file
\texttt{param.f90}. \\


%%%%%%%%%%%%%%%%%%%%%%%%
%%%   OUTPUT FILES   %%%
%%%%%%%%%%%%%%%%%%%%%%%%

\section{Output files}

At run time, the codes creates an output directory whose name should
be given in the parameter file. It then produces a series of output
files with the data from the run. There are so called \texttt{0D}
files (with extension \texttt{*.tl}), \texttt{1D} files (with
extensions \texttt{*.rl}, \texttt{*.zl} and \texttt{*.dl}), and
\texttt{1D} files (with extension \texttt{*.2D}). \\

The \texttt{OD} files refer to scalar quantities obtained from the
spatial arrays as functions of time. These scalar quantities include
the maximum (\texttt{max}), the minimum (\texttt{min}), and three
different norms of the spatial arrays: maximum absolute value
(\texttt{nm1}), root mean square (\texttt{nm2}), and total variation
(\texttt{var}). The value of different variables at the origin is also
output. \\

The \texttt{1D} files contain the complete arrays along the coordinate
axis and diagonals at different times, while the \texttt{2D} files
output the full arrays.  Beware: If you do output very often these
files can become quite large! \\

Since we can have several grid refinement levels, and grid boxes,
the file names are appended with a number corresponding to the specific
box and level (all grid levels have output). For example: \\

\texttt{alphab0l0.rl}: \hspace{5mm} Level 0 (coarsest grid) \\
\texttt{alphab0l1.rl}: \hspace{5mm} Level 1 \\
... \\

All files are written in ASCII, and using a format adapted to XGRAPH
(but other graphic packages should be able to read them).\\

Output is controlled by the following parameters:

\begin{list}{}{
\setlength{\leftmargin}{35mm}
\setlength{\labelsep}{10mm}
\setlength{\labelwidth}{20mm}}

\item[\texttt{directory}] Name of directory for output.

\item[\texttt{Ninfo}] How often do we output information to screen?

\item[\texttt{Noutput0D}] How often do we do 0D output?

\item[\texttt{Noutput1D}] How often do we do 1D output?

\item[\texttt{Noutput2D}] How often do we do 2D output?

\item[\texttt{outvars0D}] Arrays that need 0D output (a list separated by commas).

\item[\texttt{outvars1D}] Arrays that need 1D output (a list separated by commas).

\item[\texttt{outvars2D}] Arrays that need 2D output (a list separated by commas).

\end{list}

\vspace{3mm}


%%%%%%%%%%%%%%%%%%%%%%
%%%   CHECKPOINT   %%%
%%%%%%%%%%%%%%%%%%%%%%


%%%%%%%%%%%%%%%%
%%%   GRID   %%%
%%%%%%%%%%%%%%%%

\setcounter{equation}{0}
\section{Numerical grid}
\label{sec:grid}


%%%%%%%%%%%%%%%%%%%%%
%%%   SPACETIME   %%%
%%%%%%%%%%%%%%%%%%%%%

\setcounter{equation}{0}
\section{Spacetime and evolution equations}

The code uses cylindrical coordinates $(r,z)$ is space. Notice that
what the code calls $r$ is frequently called $\rho$. The distance to
the origin in the code is in fact called $rr:=(r^2 + z^2)^{1/2}$. \\

The spatial metric is written in the following way:
\begin{align}
dl^2 &= e^{4 \phi(r,z,t)} \left[ A(r,z,t) dr^2 + B(r,z,t) dz^2 
+ r^2 H(r,z,t) d\varphi^2 \right. \nonumber \\
&+ 2 \left. r \left( C(r,z,t) dr dz + r^2 C_1(r,z,t) dr d\varphi
+ r C_2(r,z,t) dz d\varphi \right) \right] \; ,
\label{eq:metric}
\end{align}
with $\phi$ the conformal factor.We also define the function $\psi =
e^\phi$ and use it instead of $\phi$ in many expressions. For the
lapse function we use the array $\alpha(r,z,t)$, while the shift in
principle has three components: $\beta^r(r,z,t)$, $\beta^z(r,z,t)$,
$\beta^\varphi(r,z,t)$. \\

Notice that when there is no angular momentum the metric can be
simplified since in that case we have $\beta^\varphi=0$, and
$C_1=C_2=0$.  The code controls this using the logical parameter
\texttt{angmom}: if this parameter is true then those arrays are
turned on, while if it is false they are off.  By default the
parameter is false, so that those arrays are turned off.


%%%%%%%%%%%%%%%%%%%%%%%%%%
%%%   REGULARIZATION   %%%
%%%%%%%%%%%%%%%%%%%%%%%%%%


%%%%%%%%%%%%%%%%%%
%%%   MATTER   %%%
%%%%%%%%%%%%%%%%%%

\setcounter{equation}{0}
\section{Matter}
\label{sec:matter}


%%%%%%%%%%%%%%%%%%%%%%%
%%%   CONSTRAINTS   %%%
%%%%%%%%%%%%%%%%%%%%%%%


%%%%%%%%%%%%%%%%%%%%%%%%%%%%%%
%%%   SLICING CONDITIONS   %%%
%%%%%%%%%%%%%%%%%%%%%%%%%%%%%%

\setcounter{equation}{0}
\section{Slicing conditions}
\label{sec:slicing}


%%%%%%%%%%%%%%%%%%%%%%%%%%%%
%%%   SHIFT CONDITIONS   %%%
%%%%%%%%%%%%%%%%%%%%%%%%%%%%

\setcounter{equation}{0}
\section{Shift conditions}
\label{sec:shift}


%%%%%%%%%%%%%%%%%%%%%%%%%%%%%%%
%%%   BOUNDARY CONDITIONS   %%%
%%%%%%%%%%%%%%%%%%%%%%%%%%%%%%%


%%%%%%%%%%%%%%%%%%%%%%%%
%%%   INITIAL DATA   %%%
%%%%%%%%%%%%%%%%%%%%%%%%

\setcounter{equation}{0}
\section{Initial data}
\label{sec:initial}

The type of initial data is controlled by the character-type parameter
\texttt{idata}.  If you add a new type of initial data it should be
appended to the list of allowed values for this parameter in the file
\texttt{src/base/param.f90}.  You should also add a corresponding call to your
initial data routine in the file \texttt{src/base/initial.f90}. \\


%%%%%%%%%%%%%%%%%%%%
%%%   HORIZONS   %%%
%%%%%%%%%%%%%%%%%%%%


%%%%%%%%%%%%%%%%%%%%%%%%%%%%%
%%%   NUMERICAL METHODS   %%%
%%%%%%%%%%%%%%%%%%%%%%%%%%%%%


%%%%%%%%%%%%%%%%%%%%%%%%%%%%
%%%   ELLIPTIC SOLVERS   %%%
%%%%%%%%%%%%%%%%%%%%%%%%%%%%


%%%%%%%%%%%%%%%%%%%%%%%%%%%%%%
%%%   LIST OF PARAMETERS   %%%
%%%%%%%%%%%%%%%%%%%%%%%%%%%%%%

\section{List of main code parameters}
\label{sec:parameters}


%%%%%%%%%%%%%%%%%%%%%%%%%%%%
%%%   EDITING THE CODE   %%%
%%%%%%%%%%%%%%%%%%%%%%%%%%%%

\section{Editing the code}
\label{sec:editing}


%%%%%%%%%%%%%%%%%%%%%%
%%%   OLLINGRAPH   %%%
%%%%%%%%%%%%%%%%%%%%%%

\section{Ollingraph}
\label{sec:ollingraph}

The code includes a simple 1D visualization package called
``ollingraph''.  It is written in Python and uses Matplotlib to plot
simple line plots and animations.  At the moment it should work fine
under python 2.7, but is seems to have problems with python 3 (I will
try to fix this later). It is supposed to have similar functionality to
the old xgraph and ygraph packages, and expects the data files in the
same format (see below). \\

The plots are quite simple, and meant for quick/dirty interactive
visualization.  These plots are not supposed to be used for figures
intended for publication (use gnuplot or something similar for
that). To use the package type: \\

\texttt{ollingraph  file1 file2 ... filen} \\

When plotting several data files at once, they are all assumed to be
of the same type.  One can also add a custom name for the plot using
the option --title: \\

\texttt{ollingraph --title="Plot name" file1 file2 ... filen} \\

If this option is not there the plot is just named after the name of
the data file being plotted, assuming there is only one, or just says
``Multiple data files'' if there is more than one file. \\

Before using it make sure that \texttt{ollingraph} has execution
permission: \\

\texttt{chmod +x ollingraph} \\

There is also a package for simple plots of 2D arrays called
\texttt{ollingraph2D}. It is called in a similar way:

\texttt{ollingraph2D  file} \\

But in this case the file to be plotted must have extension
\texttt{.2D}.

\vspace{5mm}

The data files are expected to have the following format:

\begin{itemize}

\item 0D files (those ending in .tl):

\begin{enumerate}

\item One comment line starting with \texttt{\#} or \texttt{"} with
  the file name (this line could be missing and it should still work).

\item A series of lines with the data points, with the x and y values
  separated by blank spaces.

\end{enumerate}

\item 1D evolution-type files (those ending in .rl):

\begin{enumerate}

\item Each time step begins with a comment line starting with
  \texttt{\#} or \texttt{"} that contains the time in the format: \\

\texttt{\#Time = (real number)} \\

\item A series of lines with the data points, with the x and y values
  separated by blank spaces.

\item One or more blank lines to separate the next time step.

\end{enumerate}

\end{itemize}


%%%%%%%%%%%%%%%%%%%%%%
%%%   REFERENCES   %%%
%%%%%%%%%%%%%%%%%%%%%%


%%%%%%%%%%%%%%%
%%%   END   %%%
%%%%%%%%%%%%%%%

\end{document}

